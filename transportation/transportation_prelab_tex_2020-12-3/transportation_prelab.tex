\documentclass[twoside]{article}%{combine}
%\usepackage{url}
\usepackage{../../tex/html}
\usepackage{epstopdf}
\usepackage{amsfonts,amsmath,color,amsthm,amssymb, enumerate, bbm, subfig}
\usepackage{graphicx}
%\usepackage[DIV=14,BCOR=2mm,headinclude=true,footinclude=false]{typearea}
%\usepackage[font=small,labelfont=bf]{caption}
\usepackage{hyperref}
\usepackage{tikz, etoolbox}

\usetikzlibrary{shapes}
\usetikzlibrary{arrows}
%\usepackage[margin=1in]{geometry}
\usepackage{graphicx,amsmath,gentium,tikz,caption}
\usetikzlibrary{patterns}
\usetikzlibrary{matrix,arrows,positioning,shapes}
\usetikzlibrary{arrows.meta}
\tikzset{
  a/.style={-{Stealth[scale=1.3,angle'=45]},semithick}
}
%\usepackage{xfrac,fontspec,unicode-math}
%\setmathfont[version=cambria]{Cambria Math}
%\mathversion{cambria}
\usepackage[letterpaper, portrait, margin=1.1in]{geometry}
\usepackage{amsmath,amsthm}
\usepackage{mathtools}
\newtheorem*{definition}{Definition}
\usepackage{tcolorbox}
\tcbset{colback=white,colframe=black}
\everymath{\displaystyle}

\makeatletter
\@ifundefined{namelength}{
\newlength{\namelength}
\settowidth{\namelength}{{\bf \Large Name: }}
\newlength{\namelinelength}
\setlength{\namelinelength}{\textwidth}
\addtolength{\namelinelength}{-\namelength}
}{}

\@ifundefined{vs}{
\newcommand*{\vs}[1]{\par
  \vspace*{#1\baselineskip}%
  \@afterindentfalse
  \@afterheading
}
}{}
\makeatother



\def\fancytitle#1#2#3{
      \centerline{\framebox{\framebox{ \parbox{.8\textwidth}{ \bf ENGRI 1101 \hfill
      Engineering Applications of OR \ \ \ \  Fall 2020 \hfill #3 #1 \\
\mbox{ }\hfill
      \hfill\mbox{ } \\[1mm] \mbox{ } \hfill{\Large \bf #2}\hfill
      \mbox{ }} }}}
      
\vs 2
}

\def\handout#1#2{\fancytitle{#1}{#2}{Handout}}
\def\review#1#2{\fancytitle{#1}{#2}{Review}}
\def\homework#1#2{\fancytitle{#1}{#2}{Homework}}
\def\exercises#1{\fancytitle{}{#1}{Exercises}}
\def\solution#1#2{\fancytitle{#1}{#2}{Solutions}}
\def\final#1#2{\fancytitle{#1}{#2}{Final}
      \noindent {\bf \Large Name:} \rule{\namelinelength}{0.5pt}
      \vspace*{\baselineskip}}
\def\prelim#1#2{\fancytitle{#1}{#2}{Prelim}
      \noindent {\bf \Large Name:} \rule{\namelinelength}{0.5pt}
      \vspace*{\baselineskip}}
\def\quiz#1#2{\fancytitle{#1}{#2}{Quiz}
      \noindent {\bf \Large Name:} \rule{\namelinelength}{0.5pt}
      \vspace*{\baselineskip}}
\def\lab#1#2{\fancytitle{#1}{#2}{Lab}
      \noindent {\bf \Large Name:} \rule{\namelinelength}{0.5pt}
      \vspace*{\baselineskip}}
\def\prelab#1#2{\fancytitle{#1}{#2}{Prelab}
      \noindent {\bf \Large Name:} \rule{\namelinelength}{0.5pt}
      \vspace*{\baselineskip}}

\raggedbottom

\begin{document}

\prelab{6}{ The Transportation Problem}

\noindent

Objectives:

\begin{itemize}
\item   Introduce students to the transportation problem
\item   Give an application of the
transportation problem of the ``caterer's problem''
\end{itemize}

\noindent Key Ideas:
\begin{itemize}
\item   supply point and supply constraint
\item   demand point and demand constraint
\item   balanced transportation problem
\item   bipartite graph
\item   integrality properties
\item sensitivity analysis
\end{itemize}
\textbf{Reading Assignment:}
\begin{itemize}
\item
Read the first part of Handout 7 on the transportation problem (first 4 pages)\end{itemize}

\textbf{Brief description:}
In this lab, we will
\begin{enumerate}
\item Learn how to formulate a somewhat unexpected problem as a transportation problem
\item Continue to develop our facility in using Jupyter notebooks to solve optimization problems
\end{enumerate}

\smallskip
\noindent
Read and understand the problem, and think about how you would model it as a mathematical optimization problem (i.e., what is a feasible solution of the problem? What is the objective function? How can you express the objective functions and constraints every feasible solution must satisfy mathematically?). 


\smallskip
\noindent
\textbf{The Caterer's Problem} (From [Winston]) The Carter Caterer Company must have the following number of clean napkins available at the
beginning of each of the next four days: day 1, 15; day 2, 12; day
3, 18; day 4, 6. After being used, a napkin can be cleaned by one
of two methods: fast service or slow service. Fast service costs
10 cents per napkin, and a napkin cleaned via fast service is
available for use the day after it is last used. Slow service
costs 6 cents per napkin, and these napkins can be reused two days
after they are last used. New napkins can be purchased for a cost
of 20 cents per napkin. The catering company currently has no
napkins, whatsoever. We wish to meet the demand for the next four
days as cheaply as possible.

\begin{enumerate}
\item Give one feasible solution to the problem that uses all three options of obtaining clean
napkins - buying them new, cleaning them fast, or cleaning them slowly. Compute the 
objective function value for this solution.
\item Give a lower bound on the cost of {\it any} feasible solution.
\end{enumerate}
\end{document}
